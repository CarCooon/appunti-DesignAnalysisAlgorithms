\chapter{Shortest Paths}
\label{cap:ShortestPaths}

Come abbiamo già visto, la ricerca in ampiezza (BFS) permette di trovare il cammino minimo in un grafo quando tutti gli archi sono considerati equivalenti. Tuttavia, in molte applicazioni reali questa condizione non è soddisfatta: per individuare il percorso più rapido tra due città o il tragitto più veloce per un pacchetto dati in rete, è necessario distinguere tra archi con caratteristiche diverse. Infatti, in questi contesti accade che alcune distanze tra le città saranno probabilmente molto più grandi di altre, e alcune connessioni in una rete di computer sono tipicamente molto più veloci di altre (ad esempio, alcune connessioni a bassa larghezza di banda contro connessioni ad alta velocità in fibra ottica). È quindi naturale considerare dei grafi pesatidove a ogni arco è associato un peso che ne rappresenta il costo o l'importanza.

\section{Weighted Graphs}
Un \textbf{weighted graph} (grafo pesato) è un grafo che ha un'etichetta numerica $w(e)$ associata a ciascun arco $e$, chiamata il peso (weight) dell'arco $e$. Per $e = (u,v)$, indichiamo il suo peso con la notazione $w(e) = w(u,v)$. 

\begin{figure}[!ht]
    \centering
    \includegraphics[width=0.9\textwidth]{immagini/ShortestPath/weighted_graph.png}
    \caption{Esempio di grafo pesato i cui vertici rappresentano i principali aeroporti degli Stati Uniti e i cui pesi degli archi rappresentano le distanze in miglia.}
    \label{fig:weighted_graph}
\end{figure}


\clearpage
\noindent
Sia $G$ un grafo pesato. La lunghezza (o peso) di un percorso $P$ è la somma dei pesi degli archi di $P$. Cioè, se $P = ((v_0,v_1), (v_1,v_2), \ldots , (v_{k-1},v_k))$, allora la lunghezza di $P$, indicata con $w(P)$, è definita come:

$$
w(P) = \sum_{i=1}^{k} w(v_{i-1}, v_i)
$$

\noindent
La \textbf{distanza} da un vertice $u$ a un vertice $v$ in $G$, indicata con $d(u,v)$, è la lunghezza di un percorso di lunghezza minima (chiamato anche \textbf{shortest path}) da $u$ a $v$, se tale percorso esiste.
Se non esiste alcun percorso da $u$ a $v$, allora utilizziamo la convenzione $d(u,v) = \infty$.

\paragraph{Proprietà di uno shortest path} 
\begin{enumerate}
    \item Un frammento di uno shortest path è anch'esso uno shortest path. Cioè, se $P$ è uno shortest path da $u$ a $v$ e $w$ è un vertice su $P$, allora il frammento di $P$ da $u$ a $w$ è uno shortest path da $u$ a $w$, e il frammento di $P$ da $w$ a $v$ è uno shortest path da $w$ a $v$.
    \item Gli shortest path da un vertice sorgente $s$ a tutti gli altri vertici costituiscono uno \textbf{shortest path spanning tree} (o shortest path tree) radicato in $s$.
\end{enumerate}

\begin{figure}[!ht]
    \centering
    \includegraphics[width=0.8\textwidth]{immagini/ShortestPath/shortest_path_tree.png}
    \caption{Esempio di uno shortest path tree radicato nel vertice PVD.}
    \label{fig:ex_shortest_path}
\end{figure}

\paragraph{Problemi legati agli shortest path} 
\begin{itemize}
    \item $(u, v)$-shortest path\\
    Dati due vertici $u$ e $v$, trovare lo shortest path tra $u$ e $v$.
    \item Single source shortest path\\
    Dato un vertice $s$, trovare lo shortest path tree radicato in $s$ (ovvero, trova lo shortest path da $s$ a tutti gli altri vertici).
    \item All-pairs shortest paths\\
    Trovare gli shortest path per tutte le coppie di vertici nel grafo.
\end{itemize}

\noindent
Si noti come dalla soluzione del problema single source shortest path sia possibile ottenere la soluzione del problema \emph{$(u, v)$-shortest path}, mentre dalla soluzione del problema \emph{all-pairs shortest paths} sia possibile ottenere la soluzione di entrambi gli altri problemi. Tuttavia, è anche possibile risolvere il problema \emph{all-pairs shortest paths} eseguendo l'algoritmo per il \emph{single source shortest path} per ogni vertice del grafo.

\clearpage
\section{Algoritmo di Dijkstra}
Si vuole sviluppare un \textbf{algoritmo Greedy} per la risoluzione del problema \emph{single source shortest path}. L'idea principale nell'applicare il \emph{metodo Greedy} al problema degli shortest path da una singola sorgente è quella di eseguire una "ricerca in ampiezza pesata" a partire dal vertice sorgente $s$. In particolare, possiamo utilizzare il \emph{metodo Greedy} per sviluppare un algoritmo che cresce iterativamente una "nuvola" di vertici a partire da $s$, con i vertici che entrano nella nuvola in ordine di distanza da $s$. Quindi, in ogni iterazione, il prossimo vertice scelto è il vertice al di fuori della nuvola che è più vicino a $s$. L'algoritmo termina quando non ci sono più vertici al di fuori della nuvola (o quando quelli al di fuori della nuvola non sono connessi a quelli all'interno della nuvola, e quindi hanno distanza infinita), a quel punto abbiamo uno shortest path da $s$ a ogni vertice di $G$ raggiungibile da $s$. Questo approccio è un esempio semplice, ma comunque potente, del \emph{metodo Greedy}. 

L'applicazione del \emph{metodo Greedy} al problema degli shortest path da una singola sorgente porta a un algoritmo noto come \textbf{algoritmo di Dijkstra}.


\subsection{Edge Relaxation - Greedy Choice}
Definiamo un'etichetta $D(v)$ per ogni vertice $v$ in $V$, che utilizziamo per approssimare la distanza da $s$ a $v$ in $G$. Il significato di queste etichette è che $D(v)$ memorizzerà sempre la lunghezza del miglior percorso che abbiamo trovato finora da $s$ a $v$. Difatti, possiamo pensare a $D(v)$ come un lmite superiore della distanza minima reale $d(s,v)$. Inoltre, definiamo un insieme $C$, che rappresenta l'insieme di vertici nella "nuvola", ovvero quei vertici per i quali abbiamo già calcolato la distanza minima da $s$.

Inizialmente, definiamo $D(s) = 0$ e $D(v) = \infty$ per ogni $v \neq s$, e l'insieme $C$ come l'insieme vuoto. Ad ogni iterazione dell'algoritmo, selezioniamo un vertice $u$ non in $C$ con l'etichetta $D(u)$ più piccola, e lo aggiungiamo a $C$ (Generalmente, utilizzeremo una coda di priorità per selezionare tra i vertici al di fuori della nuvola, per via dell'efficienza del metodo \texttt{remove\_min()}). 

Nella prima iterazione, ovviamente, aggiungeremo $s$ a $C$. Una volta che un nuovo vertice $u$ viene aggiunto a $C$, aggiorniamo l'etichetta $D(v)$ di ogni vertice $v$ adiacente a $u$ e che è al di fuori di $C$, per riflettere il fatto che potrebbe esserci un nuovo e migliore modo per raggiungere $v$ passando per $u$. Questa operazione di aggiornamento è nota come \emph{rilassamento degli archi} (\textbf{edge relaxation}), poiché prende una vecchia stima e verifica se può essere migliorata per avvicinarsi al suo valore reale. L'operazione specifica di rilassamento degli archi è la seguente:

\vspace{1\baselineskip}
$\quad \quad \quad \textbf{Edge Relaxation:}$
$$
\begin{aligned}
    &\textbf{if } D(u) + w(u,v) < D(v) \textbf{ then} \\
    &\quad D(v) = D(u) + w(u,v)
\end{aligned}
$$

\clearpage
\subsection{Pseudocodice ed Esempio}
L'algoritmo di Dijkstra è descritto nel seguente pseudocodice:
\vspace{1\baselineskip}
\hrule
\begin{verbatim}
Input: A weighted graph G with nonnegative edge weights, 
       and a source vertex s of G.
Output: The length of a shortest path from s to v for each 
        vertex v of G.
1.  Algorithm ShortestPath(G, s):   
2.      Initialize D[s] = 0 and D[v] = +infinity for each vertex v != s
3.      Let a priority queue Q contain all the vertices of G 
            using the D labels as keys.
4.      while Q is not empty do
5.          // pull a new vertex u into the cloud
6.          u = value returned by Q.remove_min()
7.          for each vertex v adjacent to u such that v is in Q do
8.              // perform the relaxation procedure on edge (u,v)
9.              if D[u] + w(u,v) < D[v] then
10.                 D[v] = D[u] + w(u,v)
11.                 Change to D[v] the key of vertex v in Q.
12.     return the label D[v] of each vertex v                  
\end{verbatim}
\hrule 
\vspace{1\baselineskip}

\clearpage

\vspace{2\baselineskip}
\begin{figure}[!ht]
    \centering
    \includegraphics[width=1\textwidth]{immagini/ShortestPath/ex_dijkstra1.png}
    \label{fig:ex_dijkstra1}
\end{figure}

\begin{figure}[!ht]
    \centering
    \includegraphics[width=1\textwidth]{immagini/ShortestPath/ex_dijkstra2.png}
    \caption{Esempio di esecuzione dell'algoritmo di Dijkstra. L'esempio inizia con il vertice sorgente $A$ già nella nuvola (in grigio chiaro) e con le etichette dei vertici adiacenti aggiornate. Ad ogni passo, il vertice con l'etichetta minima viene aggiunto alla nuvola e le etichette dei suoi vicini vengono aggiornate di conseguenza. Al termine dell'algoritmo, le etichette rappresentano le distanze minime da $A$ a tutti gli altri vertici. Gli archi in rosso al di fuori della nuvola rappresentano il percorso minimo attuale verso quel vertice. Gli archi in rosso nella nuvola rappresentano lo shortest path tree radicato in $A$. Gli archi tratteggiati sono quelli che non fanno parte dello shortest path tree.}
    \label{fig:ex_dijkstra2}
\end{figure}

\clearpage

\subsection{Dimostrazione della correttezza dell'algoritmo}
Potrebbe non essere immediatamente chiaro il motivo per cui l'algoritmo di Dijkstra trovi correttamente lo shortest path dal vertice di partenza $s$ a ogni altro vertice $u$ nel grafo. Perché la distanza minima da $s$ a $u$ è uguale al valore dell'etichetta $D(u)$ al momento in cui il vertice $u$ viene rimosso dalla coda di priorità $Q$ e aggiunto alla nuvola $C$? La risposta a questa domanda dipende dal fatto che \textbf{non ci siano archi con peso negativo nel grafo}, poiché ciò consente al metodo Greedy di funzionare correttamente.

\paragraph{Teorema:} Nell'algoritmo di Dijkstra, ogni volta che un vertice $v$ viene inserito nella soluzione $C$, l'etichetta $D(v)$ è uguale a $d(s, v)$, la lunghezza del percorso minimo da $s$ a $v$.

\paragraph{Dimostrazione:} Per dimostrare la tesi, procediamo per assurdo. Ipotizziamo che l'algoritmo sbagli per qualche vertice. Sia $z$ il \textbf{primo vertice} che l'algoritmo inserisce in $C$ tale che la sua etichetta sia errata, ovvero:
$$ D(z) > d(s, z) $$
Poiché $z$ è il \textit{primo} errore, significa che per tutti i vertici inseriti in $C$ prima di $z$, l'algoritmo ha calcolato la distanza corretta.

Esiste sicuramente un percorso minimo reale da $s$ a $z$ (altrimenti la distanza sarebbe $\infty$). Chiamiamo questo percorso $P$.
Consideriamo il momento esatto in cui l'algoritmo sta per inserire $z$ in $C$. Analizziamo il percorso $P$ partendo da $s$:
\begin{itemize}
    \item Il vertice sorgente $s$ appartiene già a $C$.
    \item Il vertice $z$ non appartiene a $C$.
    \item Di conseguenza, percorrendo il cammino $P$ da $s$ a $z$, deve necessariamente esistere un primo arco $(x, y)$ che attraversa il confine di $C$, tale che $x \in C$ (il vertice "dentro") e $y \notin C$ (il primo vertice "fuori").
\end{itemize}
Dunque, sia $y$ il \textbf{primo vertice} di $P$ che non appartiene a $C$, e sia $x$ il predecessore di $y$ lungo il percorso $P$. 
\begin{itemize}
    \item Poiché $y$ è il primo vertice fuori da $C$, ne consegue che $x$ deve trovarsi dentro $C$ (al limite $x$ potrebbe coincidere con $s$).
\end{itemize}


\clearpage
\begin{figure}[!ht]
    \centering
    \includegraphics[width=1\textwidth]{immagini/ShortestPath/dimostrazione.png}
    \caption{Rappresentazione grafica della dimostrazione della correttezza dell'algoritmo di Dijkstra.}
    \label{fig:dijkstra_proof}
\end{figure}

\noindent
Poiché $z$ è stato definito come il \textit{primo} vertice "sbagliato" aggiunto a $C$, e $x$ è stato aggiunto a $C$ prima di $z$, allora l'etichetta di $x$ è sicuramente corretta:
$$ D(x) = d(s, x) $$

\noindent
Quando $x$ è stato inserito in $C$, l'algoritmo ha eseguito la procedura di \textit{Edge Relaxation} sui suoi vertici adiacenti, incluso $y$. Questo aggiornamento garantisce che:
$$ D(y) \le D(x) + w(x, y) = d(s, x) + w(x, y) $$

\noindent
Poiché $x$ e $y$ sono vertici consecutivi sul percorso minimo $P$ (che è un percorso minimo globale), anche il sotto-percorso da $s$ a $y$ deve essere un percorso minimo. Pertanto:
$$ d(s, y) = d(s, x) + w(x, y) $$

\noindent
Combinando le due equazioni precedenti, otteniamo che l'etichetta di $y$ è già corretta al momento dell'estrazione di $z$:
$$ D(y) = d(s, y) $$

\noindent
Ora arriviamo al cuore della dimostrazione logica.
L'algoritmo di Dijkstra è \textit{greedy}: seleziona sempre il vertice fuori da $C$ con l'etichetta $D$ più piccola. In questo momento, sia $z$ che $y$ sono fuori da $C$ (sono nella coda $Q$), ma l'algoritmo ha scelto di estrarre $z$. Questo implica necessariamente che:
$$ D(z) \le D(y) $$

\noindent
Qui entra in gioco l'ipotesi fondamentale che non esistano archi con peso negativo. La distanza reale da $s$ a $z$ può essere scomposta nella distanza da $s$ a $y$ più la distanza rimanente da $y$ a $z$:
$$ d(s, z) = d(s, y) + d(y, z) $$

\noindent
Poiché i pesi sono non negativi, allora si ha che $d(y, z) \ge 0$. Di conseguenza:
$$ d(s, y) \le d(s, z) $$

\noindent
Mettiamo insieme la catena di disuguaglianze:
\begin{alignat*}{4}
    D(z)    &\le{}& D(y)    &\quad& \text{(} &\text{per la scelta greedy dell'algoritmo)} \\
    D(y)    &={} & d(s,y)  &\quad& \text{(} &\text{perché l'etichetta di $y$ è già corretta al momento dell'estrazione di $z$)} \\
    d(s,y)  &\le{}& d(s,z) &\quad& \text{(} &\text{perché i pesi sono non-negativi)}
\end{alignat*}

\noindent
Unendo tutto:
$$ D(z) \le d(s, z) $$

\noindent
Ma questo contraddice la nostra ipotesi iniziale per assurdo, secondo cui $D(z) > d(s, z)$.

\vspace{1\baselineskip}
\noindent
La contraddizione dimostra che non può esistere un "primo vertice sbagliato" $z$. Pertanto, per ogni vertice $v$ aggiunto a $C$, l'etichetta $D(v)$ rappresenta sempre la distanza minima corretta.



