% --- PREAMBOLO ---

% Definisce il tipo di documento (articolo, report, book, letter, ecc.)
% a4paper imposta la pagina in formato A4, 12pt la dimensione del font
\documentclass[a4paper, 11pt]{book}

% --- Font Times New Roman e lingua---
\usepackage[T1]{fontenc}
\usepackage[utf8]{inputenc}
\usepackage[italian]{babel}
\usepackage[normalem]{ulem}

\usepackage{newtxtext,newtxmath}
\usepackage{microtype}

% --- STILE DELL'INDICE ---
\makeatletter
% capitolo, livello 0: indentazione 0em, larghezza numero 1.5em
\renewcommand*\l@chapter{\@dottedtocline{0}{0em}{1.5em}}
% sezione, livello 1: indentazione 1.5em, larghezza numero 2.3em
\renewcommand*\l@section{\@dottedtocline{1}{1.5em}{2.3em}}
% sottosezione, livello 2: indentazione 3.8em, larghezza numero 3.2em
\renewcommand*\l@subsection{\@dottedtocline{2}{3.8em}{3.2em}}
\makeatother



% rinomina "Indice" in "Sommario"
\addto\captionsitalian{
  \renewcommand{\contentsname}{Sommario}
}



% --- Impaginazione ---
\usepackage[margin=3cm,bindingoffset=0.5cm]{geometry}
\usepackage{setspace}
\setstretch{1.2}
\setlength{\parindent}{1cm}

% --- Figure, tabelle ---
\usepackage{graphicx}
\graphicspath{{./}{./immagini/}} % cerca immagini anche in ./immagini
\usepackage{float}
\usepackage{subcaption}
\usepackage{tabularx,booktabs}
\usepackage{enumitem}
\usepackage{amsmath}


% --- Bibliografia ---
\usepackage[backend=biber,sorting=none,style=numeric]{biblatex}
\addbibresource{bib.bib}
\usepackage{csquotes}

% --- Numerazione e intestazioni ---
\usepackage{fancyhdr}
\pagestyle{fancy}
\fancyhf{}
\fancyfoot[C]{\thepage}
\renewcommand{\headrulewidth}{0pt}
\renewcommand{\footrulewidth}{0pt}
\fancypagestyle{plain}{
  \fancyhf{}
  \fancyfoot[C]{\thepage}
  \renewcommand{\headrulewidth}{0pt}
  \renewcommand{\footrulewidth}{0pt}
}

\usepackage[bottom]{footmisc}
\setlength{\skip\footins}{8pt plus 2pt minus 1pt}

% --- Link e riferimenti ---
\usepackage[hidelinks]{hyperref}
\usepackage{cleveref}

% --- INFORMAZIONI SUL DOCUMENTO ---
\title{Alberi}
\author{Carmine}
\date{\today} % \today inserisce la data di compilazione. Puoi anche scrivere una data fissa, es. "Ottobre 2025"



% --- CORPO DEL DOCUMENTO ---
\begin{document}

% ================= FRONTESPIZIO =================
\begin{titlepage}
    \begin{center}
        \huge{\uppercase{Università degli Studi di Salerno}}\\[10mm]
        \uppercase{\normalsize DIPARTIMENTO DI INGEGNERIA DELL'INFORMAZIONE ED ELETTRICA\\ E MATEMATICA APPLICATA}\\[15mm]
        \normalsize{Corso: \\DESIGN AND ANALYSIS OF ALGORITHMS}\\[20mm]
        \includegraphics[width=0.35\textwidth]{immagini/logo_unisa}\\[20mm]
        \textbf{\large APPUNTI}\\[10mm]
    \end{center}

    \begin{center}
        \Large{\textbf{Carmine Terracciano}}\\[2mm]
        \large{Mat. IE22700109}\\[10mm]
    \end{center}

    \vfill
    \begin{center}
        \large \uppercase{Anno Accademico 2025/2026}
    \end{center}
\end{titlepage}

% Crea un sommario/indice
\tableofcontents


% --- INIZIO DEL CONTENUTO ---
\cleardoublepage
\input{capitoli/cap1_BST}

\cleardoublepage
\input{capitoli/cap2_AVL}

\cleardoublepage
\input{capitoli/cap3_MWST}

\cleardoublepage
\input{capitoli/cap4_RBT}

\cleardoublepage
\input{capitoli/cap5_HashTables}

\cleardoublepage
\input{capitoli/cap6_PriorityQueues}


% Questo comando segna la fine del documento
\end{document}